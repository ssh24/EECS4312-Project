% !TEX encoding = UTF-8
%Koma article
\documentclass[fontsize=12pt,paper=letter,twoside]{scrartcl}

%Standard Pre-amble
\usepackage[top=4cm,bottom=4cm,left=3cm,right=3cm,asymmetric]{geometry}
%\geometry{landscape}                % Activate for for rotated page geometry
%\usepackage[parfill]{parskip}    % Begin paragraphs with an empty line rather than an indent
\usepackage[table,xcdraw]{xcolor}
\usepackage{graphicx}

\usepackage{amsmath}
\usepackage{amssymb}
\usepackage{epstopdf}
\DeclareGraphicsRule{.tif}{png}{.png}{`convert #1 `dirname #1`/`basename #1 .tif`.png}
% Listings needs package courier
\usepackage{listings} % Needs 
\usepackage{courier}

\usepackage[framemethod=TikZ]{mdframed}
\usepackage{url}

\usepackage{sty/bsymb} %% Event-B symbols
\usepackage{sty/eventB} %% REQ and ENV
\usepackage{sty/calculation}

%Maths
\usepackage{amssymb,amsmath}
\def\Fl{\mathbb{F}}
\def\Rl{\mathbb{R}}
\def\Nl{\mathbb{N}}
\def\Bl{\mathbb{B}}
\def\St{\mathbb{S}}
\newcommand{\ovr}{\upharpoonright}
\newcommand{\var}[1]{\textit{#1}}
%Useful definitions
\newcommand{\mv}[1]{\textit{m\_#1}}
\newcommand{\cv}[1]{\textit{c\_#1}}
\newcommand{\degree}[1]{^{\circ}\mathrm{#1}}
%\newcommand{\comment}[1]{{\footnotesize \quad\texttt{--}\textrm{#1}}}
\newcommand{\im}[1]{i\texttt{-\!#1}}

\usepackage[headsepline]{scrpage2}
\pagestyle{scrheadings}
\ihead[]{\small EECS4312 Report1}
\ohead[]{\small \thepage}
\cfoot[]{}
\ofoot[]{}


%%%%PVS environment%%%%%%%%%%%%%%%%%%%
\lstnewenvironment{pvs}[1][]
    {\lstset{#1,captionpos=b,language=pvs,
    mathescape=true,
    basicstyle=\small\ttfamily,
    numbers=none,
    frame=single,
    % numberstyle=\tiny\color{gray},
    % backgroundcolor=\color{lightgray},
    firstnumber=auto
    }}
    {}
 %%%%%%%%%%%%%%%%%%%%%%%%%%%%%%%%
 
%%%%Verbatim environment%%%%%%%%%%%%%%%%%%%
\lstnewenvironment{code}[1][]
    {\lstset{#1,captionpos=b,
    mathescape=true,
    basicstyle=\small\ttfamily,
    numbers=none,
    frame=single,
    % numberstyle=\tiny\color{gray},
    % backgroundcolor=\color{lightgray},
    firstnumber=auto
    }}
    {}

% \newenvironment{boxed}[1]
%    {\begin{center}
%    #1\\[1ex]
%    \begin{tabular}{|p{0.9\textwidth}|}
%    \hline\\
%    }
%    { 
%    \\\\\hline
%    \end{tabular} 
%    \end{center}
%    }
 %%%%%%%%%%%%%%%%%%%%%%%%%%%%%%%%
 
 %Text in a box
\newenvironment{textbox}
    {\begin{center}
    \begin{tabular}{|p{0.9\textwidth}|}
    \hline\\
    }
    { 
    \\\\\hline
    \end{tabular} 
    \end{center}
    }

\usepackage{hyperref}

%Highlight \hl{}
\usepackage{soul}

\usepackage{enumitem}
\newlist{mylist}{itemize}{1}
\setlist[mylist]{label=\textbullet,leftmargin=1cm,nosep}

\usepackage{multirow}

% Reduce space between figure and caption
%\usepackage{caption}
%\captionsetup[table]{font=small,skip=0pt}     %% Adjust here
%or equivalently 
\usepackage[font=small,skip=4pt]{caption}
%Useful definitions
%\newcommand{\mv}[1]{\textit{m\_#1}}
%\newcommand{\cv}[1]{\textit{c\_#1}}
%\newcommand{\degree}[1]{^{\circ}\mathrm{#1}}
%\newcommand{\comment}[1]{{\footnotesize \quad\texttt{--}\textrm{#1}}}

% Set the header
\ihead[]{\small EECS4312 Isolette Assignment}


%%%%%%%%%%%%Enter your names here%%%%%%%%
\author{\textbf{Juan Loja (lojag95@cse.yorku.ca)}
\and \textbf{Sadman Sakib Hasan (cse23152@cse.yorku.ca)}
}
%%%%%%%%%%%%%%%%%%%%%%%%%%%%%%%%

\date{\today} % Display a given date or no date

\begin{document}
\title{EECS4312 Project\\Nuclear Waste Tracker Project}
\maketitle

\noindent \textbf{Prism account used for submission}: cse23152@cse.yorku.ca

\begin {mdframed}
\textbf{\copyright This document is not for public distribution}. This document may only be used by EECS4312 students registered at York University. By downloading this document from the department, registered York students agree to keep this document (and all documents associated with assignments, projects or laboratories) private for their personal use, and may not communicate it to anyone else. 

Students must obey York regulations on academic honesty requiring that students do the work of the Lab on their own, and not cheat by sharing with others or using and/or submitting the work of others. If you use \textit{github} or similar repository for your work,  the repository must be private. Placing your work in the public domain infringes on academic integrity. Github offers unlimited private repositories to students: \url{https://education.github.com/pack}.

\end {mdframed}

\newpage

\vspace*{2in}
\begin{center}
\huge{\textbf{Requirements Document}:\\ Nuclear Waste Tracker}
\end{center}

\bigskip\bigskip

\section*{Revisions}

%%%%%%%%%%%%Table of revisions%%%%%%%%
\begin{tabular}{|l|l|p{3in}|}
\hline
Date & Revision& Description \\ 
\hline
%14 November  2017
%& 1.0       
%& Initial Re\\ 
\hline
\hline
\end{tabular}
%%%%%%%%%%%%%%%%%%%%%%%%%%%%%%%%

\newpage

%%%%%%%%%%%%%%%%%%%%%%%%%%%%%%%
\tableofcontents
\listoffigures
\listoftables
\newpage

%%%%Rest of your document goes here%%%%%%%%%%%%%%%%%%%

\section{System Overview}

A tracker system monitors the position of waste products in nuclear plants and ensures their safe handling. Our customer requires a software system that operators use to manage safe tracking of radioactive waste in their various nuclear plants. We have so far elicited the following information from our customer.

Containers of material pass through various stages of processing in the tracking part of the nuclear plant. The tracking plant consist of a number of phases usually corresponding to the physical processes that handle the radioactive materials. Not all plants have precisely the same phases.

As an example, containers (containing a a possibly radioactive material type) might arrive at an initial unpacking phase where they are stored for further processing depending on their material contents. All nuclear plants have only the following types of material: \textit{glass}, \textit{metal}, \textit{plastic}, or \textit{liquid}. No other materials are tracked.

A subsequent phase might be called the ``assay” phase to measure the recoverable material content of each container before passing onto the next phase. A next stage might be a compacting phase. A compacting phase might involve dissolving metal contents or crushing glass. Not all material types can necessarily be handled in a phase. For example, we should not move containers with liquid into a compacting phase. Finally the products of the process might be placed in storage. There may be other phases in a particular instance of the tracker.

Each container has a unique identifier and contains only one type of material. It is labelled with a preliminary radiation count (in \textit{mSv}). When a container is registered in the system, it is also placed in a phase (not necessarily an initial phase).

The sievert (symbol: Sv) is a unit of ionizing radiation dose in the International System of Units (SI) and is a measure of the health effect of radiation on the human body. Quantities that are measured in sieverts are intended to represent the stochastic health risk, which for radiation dose assessment is defined as the probability of cancer induction and genetic damage. One sievert carries with it a 5.5\% chance of eventually developing cancer.\footnote{\url{https://en.wikipedia.org/wiki/Sievert}}

For a given plant, there is an initial setup of two important fixed global parameters for a given plant: there is a limit on the maximum radiation in any phase of the plant (in units of mSv), and there is also a limit on the maximum radiation that any container in the plant may have (in mSv). An error status message shall be signalled if there is an attempt to register a new container in the system with radiation that exceeds the container limit.

Another operation is to add a new phase (this is information provided by the Domain experts). Requirements elicitation so far yields that a new phase is specified by a phase ID, a name (e.g. “compacting”), a limit on the maximum number of containers in the phase, and a list of material types that may be treated in the phase. A phase may also be removed if there are no containers anywhere in the system. Also, it is possible for an operator to move a container from one phase to another.

Obviously when dealing with dangerous materials is very important to ensure that no material goes missing and that care is taken to avoid too much material getting into a phase, in case there is a buildup of dangerous substances in one area. The tracking manager is responsible for giving permission to movements of containers between processing phases in order to avoid dangerous situations.

\newpage
\section{Abstract Grammar}




\newpage
\section{Use Cases}

\subsection{Use Case Textual Description}

\subsection{Use Case Diagram}


\newpage
\section{Acceptance Tests}

\newpage
\section{Safety Invariant}

\end{document}  